\section{Quternion表現のKalmanFilter} \label{sec:quaternion_kalman_filter}

\subsection{ジャイロのバイアスを含めたEKF}
\subsubsection{関係式}
角速度からクォータニオンを更新する式
\begin{equation}
  \dot{\tilde{\boldsymbol{q}}}_{t} =
  \frac{1}{2}
  \begin{pmatrix}
    0        & -\omega_x & -\omega_y & -\omega_z\\
    \omega_x &         0 &  \omega_z & -\omega_y\\
    \omega_y & -\omega_z &         0 &  \omega_x\\
    \omega_z &  \omega_y & -\omega_x &         0
  \end{pmatrix}
  \begin{pmatrix}
    q_0\\
    q_1\\
    q_2\\
    q_3
  \end{pmatrix} =
  \frac{1}{2}
  \boldsymbol{\Omega}(\boldsymbol{\omega})
  \tilde{\boldsymbol{q}}_{t}
\end{equation}
クォータニオンから重力加速度を求める式
\begin{equation}
  \begin{pmatrix}
    acc_x\\
    acc_y\\
    acc_z
  \end{pmatrix} =
  {}^{W}_{A}\boldsymbol{R}^{T} \boldsymbol{g}^{W}
  =
  \begin{pmatrix}
    2\left(- q_0 q_2 + q_1 q_3\right)\\
    2\left(q_0 q_1 + q_2 q_3\right)\\
    q_0^2-q_1^2-q_2^2+q_3^2
  \end{pmatrix}
  g
\end{equation}
状態量
\begin{equation}
  \boldsymbol{x} =
  \begin{pmatrix}
    q_0\\
    q_1\\
    q_2\\
    q_3\\
    b_{\omega_x}\\
    b_{\omega_y}\\
    b_{\omega_z}
  \end{pmatrix} =
  \begin{pmatrix}
    \tilde{\boldsymbol{q}}\\
    \boldsymbol{b}_{\omega}
  \end{pmatrix}
\end{equation}


\subsubsection{中身}
\begin{itemize}
  \item $\boldsymbol{x}_{k}^{+}$
  \item $\boldsymbol{P}_{k}^{+}$
\end{itemize}
がすでに求まっている状態で,
\begin{itemize}
  \item $\boldsymbol{\omega}_{k+1}$
  \item $\boldsymbol{acc}_{k+1}$
\end{itemize}
を取得したところから始まる.

謎だけど入力$\boldsymbol{u}$をジャイロだと考えて,
Predicted state estimateは
\newcommand*{\omegawithbias}[1]{
  \frac{\omega_{{k+1}_#1} - b_{\omega_{{k}_#1}}^{+}}{2}\Delta t
}
\newcommand*{\tmpbibun}[1]{
  \frac{\Delta t q_{#1}^{+}}{2}
}
\begin{align}
  &\boldsymbol{x}_{k+1}^{-} =
  \begin{pmatrix}
    \tilde{\boldsymbol{q}}_{k+1}^{-}\\
    \boldsymbol{b}_{{\omega}_{k+1}}^{-}
  \end{pmatrix} = f(\boldsymbol{x}_{k}^{+}, \boldsymbol{u}_{k+1})\\
  &=
  \begin{pmatrix}
    \frac{1}{2}
    \boldsymbol{\Omega}(\boldsymbol{\omega}_{k+1} - \boldsymbol{b}_{{\omega}_{k}}^{+})
    \tilde{\boldsymbol{q}}_{k}^{+}
    \Delta t +
    \tilde{\boldsymbol{q}}_{k}^{+}\\
    \boldsymbol{b}_{{\omega}_{k}}^{+}
  \end{pmatrix}\\
  &= {\footnotesize
    \setlength\arraycolsep{1pt}
    \begin{pmatrix}
      q_0^{+} -\omegawithbias{x}q_1^{+} -\omegawithbias{y}q_2^{+} -\omegawithbias{z}q_3^{+}\\
      \omegawithbias{x} q_0^{+} +         q_1^{+} +  \omegawithbias{z}q_2^{+} -\omegawithbias{y}q_3^{+}\\
      \omegawithbias{y} q_0^{+} -\omegawithbias{z}q_1^{+} +         q_2^{+} + \omegawithbias{x}q_3^{+}\\
      \omegawithbias{z} q_0^{+} +\omegawithbias{y}q_1^{+} -\omegawithbias{x}q_2^{+} +         q_3^{+}\\
      b_{\omega_{k_x}}^{+}\\
      b_{\omega_{k_y}}^{+}\\
      b_{\omega_{k_z}}^{+}
    \end{pmatrix}
  }
\end{align}
となるので,
\begin{align}
  &\boldsymbol{F}_{k} = \left. \frac{\partial f}{\partial \boldsymbol{x}}\right|_{\boldsymbol{x}_{k}^{+}, \boldsymbol{u}_{k+1}}\\
  &=
  {\tiny
    \setlength\arraycolsep{1pt}
    \left(
    \begin{array}{cccc}
      1 & -\omegawithbias{x} & -\omegawithbias{y} & -\omegawithbias{z}\\
      \omegawithbias{x} & 1 & \omegawithbias{z} & -\omegawithbias{y}\\
      \omegawithbias{y} & -\omegawithbias{z} & 1 &\omegawithbias{x}\\
      \omegawithbias{z} & \omegawithbias{y} & -\omegawithbias{x} & 1\\
      0 & 0 & 0 & 0\\
      0 & 0 & 0 & 0\\
      0 & 0 & 0 & 0
    \end{array}\nonumber
    \right.
  }\\
  & \left.
  \begin{array}{ccc}
     \tmpbibun{1} &  \tmpbibun{2} &  \tmpbibun{3}\\
    -\tmpbibun{0} &  \tmpbibun{3} & -\tmpbibun{2}\\
    -\tmpbibun{3} & -\tmpbibun{0} &  \tmpbibun{1}\\
     \tmpbibun{2} & -\tmpbibun{1} & -\tmpbibun{0}\\
     1 & 0 & 0\\
     0 & 1 & 0\\
     0 & 0 & 1
  \end{array}
  \right)\\
  &=
  \begin{pmatrix}
    \Box & \triangle\\
    \boldsymbol{O} & \boldsymbol{E}
  \end{pmatrix}
\end{align}
よってPredicted covariance estimateは
\begin{align}
  \boldsymbol{P}_{k+1}^{-} &= \boldsymbol{F}_{k} \boldsymbol{P}_{k+1}^{+} \boldsymbol{F}_{k}^{T} +
  \boldsymbol{Q}_{k+1}\\
  &=
  \begin{pmatrix}
    \Box & \triangle\\
    \boldsymbol{O} & \boldsymbol{E}
  \end{pmatrix}
  \begin{pmatrix}
    \clubsuit & \heartsuit\\
    \diamondsuit & \spadesuit
  \end{pmatrix}
  \begin{pmatrix}
    \Box^{T} & \boldsymbol{O}\\
    \triangle^{T} & \boldsymbol{E}
  \end{pmatrix} +
  \boldsymbol{Q}_{k+1}\\
  &=
  {\tiny
    \begin{pmatrix}
      \left(\Box \clubsuit + \triangle \diamondsuit \right) \Box^{T} + \left(\Box \heartsuit + \triangle \spadesuit \right) \triangle^{T} & \Box \heartsuit + \triangle \spadesuit\\
      \diamondsuit \Box^{T} + \spadesuit \triangle^{T} & \spadesuit
    \end{pmatrix}
  } +
  \boldsymbol{Q}_{k+1}\\
  &=
  {\tiny
    \begin{pmatrix}
      \left(\Box \clubsuit + \triangle \diamondsuit \right) \Box^{T} + \left(\Box \heartsuit + \triangle \spadesuit \right) \triangle^{T} + \boldsymbol{Q}_{k+1}^{upper} & \Box \heartsuit + \triangle \spadesuit\\
      \diamondsuit \Box^{T} + \spadesuit \triangle^{T} & \spadesuit + \boldsymbol{Q}_{k+1}^{lower}
    \end{pmatrix}
  }\\
  &=
  \begin{pmatrix}
    \clubsuit^{'} & \heartsuit^{'}\\
    \diamondsuit^{'} & \spadesuit^{'}
  \end{pmatrix}
\end{align}
注目は$\boldsymbol{F}_{k} \boldsymbol{P}_{k+1}^{+} \boldsymbol{F}_{k}^{T}$の右下$\spadesuit$が単調増加になってしまっているところ.
これで予測の部分は終わり.
ちなみに,$\boldsymbol{x}_{k+1}^{-}$はクォータニオンの要件である大きさ1を満たしていない可能性があるので,正規化する.

次に,修正の部分.
\begin{equation}
  h(\boldsymbol{x}_{k+1}^{-}) = \tilde{\boldsymbol{q}}_{k+1}^{-\ast} \boldsymbol{g}^{W} \tilde{\boldsymbol{q}}_{k+1}^{-} =
  \begin{pmatrix}
    2\left(- q_0^{-} q_2^{-} + q_1^{-} q_3^{-}\right)\\
    2\left(q_0^{-} q_1^{-} + q_2^{-} q_3^{-}\right)\\
    {q_0^{-}}^2 - {q_1^{-}}^2 - {q_2^{-}}^2 + {q_3^{-}}^2
  \end{pmatrix}
  g
\end{equation}
より,
\begin{align}
  \boldsymbol{H}_{k+1} &= \left. \frac{\partial h}{\partial \boldsymbol{x}}\right|_{\boldsymbol{x}_{k+1}^{-}}\\
  &=
  2g
  \begin{pmatrix}
    -q_2^{-} &  q_3^{-} & -q_0^{-} &  q_1^{-} & 0 & 0 & 0\\
     q_1^{-} &  q_0^{-} &  q_3^{-} &  q_2^{-} & 0 & 0 & 0\\
     q_0^{-} & -q_1^{-} & -q_2^{-} &  q_3^{-} & 0 & 0 & 0
  \end{pmatrix}\\
  &=
  \begin{pmatrix}
    \pounds & \boldsymbol{O}
  \end{pmatrix}
\end{align}
となる.
よって
\begin{align}
  \boldsymbol{y}_{k+1} &= \boldsymbol{acc}_{k+1} - h(\boldsymbol{x}_{k+1}^{-})\\
  \boldsymbol{S}_{k+1} &= \boldsymbol{H}_{k+1} \boldsymbol{P}_{k+1}^{-} \boldsymbol{H}_{k+1}^{T} + \boldsymbol{R}_{k+1}\\
  \boldsymbol{K}_{k+1} &= \boldsymbol{P}_{k+1}^{-} \boldsymbol{H}_{k+1}^{T} \boldsymbol{S}_{k+1}^{-1}
\end{align}
となって
\begin{align}
  \boldsymbol{x}_{k+1}^{+} &= \boldsymbol{x}_{k+1}^{-} + \boldsymbol{K}_{k+1} \boldsymbol{y}_{k+1}\\
  \boldsymbol{P}_{k+1}^{+} &= \left(\boldsymbol{E} - \boldsymbol{K}_{k+1} \boldsymbol{H}_{k+1}\right) \boldsymbol{P}_{k+1}^{-}
\end{align}
となる.
$\boldsymbol{R}_{k+1}$が一定であれば,
$\boldsymbol{P}_{k+1}^{+}$は今回の観測による影響は受けないことに注意.
$\boldsymbol{H}_{k+1}$経由で次のサイクルから入る.

また,
\begin{align}
  \boldsymbol{S}_{k+1} &= \boldsymbol{H}_{k+1} \boldsymbol{P}_{k+1}^{-} \boldsymbol{H}_{k+1}^{T} + \boldsymbol{R}_{k+1}\\
  &=
  \begin{pmatrix}
    \pounds & \boldsymbol{O}
  \end{pmatrix}
  \begin{pmatrix}
    \clubsuit^{'} & \heartsuit^{'}\\
    \diamondsuit^{'} & \spadesuit^{'}
  \end{pmatrix}
  \begin{pmatrix}
    \pounds & \boldsymbol{O}
  \end{pmatrix}^{T} + \boldsymbol{R}_{k+1}\\
  &=
  \pounds \clubsuit^{'} \pounds^{T} + \boldsymbol{R}_{k+1}
\end{align}
より
\begin{align}
  \boldsymbol{K}_{k+1} &= \boldsymbol{P}_{k+1}^{-} \boldsymbol{H}_{k+1}^{T} \boldsymbol{S}_{k+1}^{-1}\\
  &=
  \begin{pmatrix}
    \clubsuit^{'} & \heartsuit^{'}\\
    \diamondsuit^{'} & \spadesuit^{'}
  \end{pmatrix}
  \begin{pmatrix}
    \pounds & \boldsymbol{O}
  \end{pmatrix}^{T}
  \left(\pounds \clubsuit^{'} \pounds^{T} + \boldsymbol{R}_{k+1}\right)^{-1}\\
  &=
  \begin{pmatrix}
    \clubsuit^{'} \pounds^{T}\\
    \diamondsuit^{'} \pounds^{T}
  \end{pmatrix}
  \left(\pounds \clubsuit^{'} \pounds^{T} + \boldsymbol{R}_{k+1}\right)^{-1}\\  &=
  \begin{pmatrix}
    \clubsuit^{'} \pounds^{T} \left(\pounds \clubsuit^{'} \pounds^{T} + \boldsymbol{R}_{k+1}\right)^{-1}\\
    \diamondsuit^{'} \pounds^{T} \left(\pounds \clubsuit^{'} \pounds^{T} + \boldsymbol{R}_{k+1}\right)^{-1}
  \end{pmatrix}\\
  \boldsymbol{x}_{k+1}^{+} &= \boldsymbol{x}_{k+1}^{-} + \boldsymbol{K}_{k+1} \boldsymbol{y}_{k+1}\\
  &=
  \begin{pmatrix}
    \tilde{\boldsymbol{q}}_{k+1}^{+}\\
    \boldsymbol{b}_{{\omega}_{k+1}}^{+}
  \end{pmatrix}\\
  &=
  \begin{pmatrix}
    \tilde{\boldsymbol{q}}_{k+1}^{-}\\
    \boldsymbol{b}_{{\omega}_{k+1}}^{-}
  \end{pmatrix} +
  \begin{pmatrix}
    \clubsuit^{'} \pounds^{T} \left(\pounds \clubsuit^{'} \pounds^{T} + \boldsymbol{R}_{k+1}\right)^{-1} \boldsymbol{y}_{k+1}\\
    \diamondsuit^{'} \pounds^{T} \left(\pounds \clubsuit^{'} \pounds^{T} + \boldsymbol{R}_{k+1}\right)^{-1} \boldsymbol{y}_{k+1}
  \end{pmatrix}\\
  \boldsymbol{P}_{k+1}^{+} &= \left(\boldsymbol{E} - \boldsymbol{K}_{k+1} \boldsymbol{H}_{k+1}\right) \boldsymbol{P}_{k+1}^{-}\\
  &=
  {\footnotesize
    \left(\boldsymbol{E} -
    \begin{pmatrix}
      \clubsuit^{'} \pounds^{T} \left(\pounds \clubsuit^{'} \pounds^{T} + \boldsymbol{R}_{k+1}\right)^{-1} \pounds & \boldsymbol{O}\\
      \diamondsuit^{'} \pounds^{T} \left(\pounds \clubsuit^{'} \pounds^{T} + \boldsymbol{R}_{k+1}\right)^{-1} \pounds & \boldsymbol{O}
    \end{pmatrix}
    \right) \boldsymbol{P}_{k+1}^{-}}\\
  &=
  \begin{pmatrix}
    \boldsymbol{E} - \clubsuit^{'} \pounds^{T} \left(\pounds \clubsuit^{'} \pounds^{T} + \boldsymbol{R}_{k+1}\right)^{-1} \pounds & \boldsymbol{O}\\
    - \diamondsuit^{'} \pounds^{T} \left(\pounds \clubsuit^{'} \pounds^{T} + \boldsymbol{R}_{k+1}\right)^{-1} \pounds & \boldsymbol{E}
  \end{pmatrix}
  \boldsymbol{P}_{k+1}^{-}\\
  &=
  {\tiny
    \begin{pmatrix}
      \boldsymbol{E} - \clubsuit^{'} \pounds^{T} \left(\pounds \clubsuit^{'} \pounds^{T} + \boldsymbol{R}_{k+1}\right)^{-1} \pounds & \boldsymbol{O}\\
      - \diamondsuit^{'} \pounds^{T} \left(\pounds \clubsuit^{'} \pounds^{T} + \boldsymbol{R}_{k+1}\right)^{-1} \pounds & \boldsymbol{E}
    \end{pmatrix}
    \begin{pmatrix}
      \clubsuit^{'} & \heartsuit^{'}\\
      \diamondsuit^{'} & \spadesuit^{'}
    \end{pmatrix}
  }\\
  &=
  {\tiny
    \begin{pmatrix}
      \clubsuit^{'} - \clubsuit^{'} \pounds^{T} \left(\pounds \clubsuit^{'} \pounds^{T} + \boldsymbol{R}_{k+1}\right)^{-1} \pounds \clubsuit^{'} & \heartsuit^{'} - \clubsuit^{'} \pounds^{T} \left(\pounds \clubsuit^{'} \pounds^{T} + \boldsymbol{R}_{k+1}\right)^{-1} \pounds \heartsuit^{'}\\
      \diamondsuit^{'} - \diamondsuit^{'} \pounds^{T} \left(\pounds \clubsuit^{'} \pounds^{T} + \boldsymbol{R}_{k+1}\right)^{-1} \pounds \clubsuit^{'} & \spadesuit^{'} - \diamondsuit^{'} \pounds^{T} \left(\pounds \clubsuit^{'} \pounds^{T} + \boldsymbol{R}_{k+1}\right)^{-1} \pounds \heartsuit^{'}
    \end{pmatrix}
  }
\end{align}

さらに,更新の時と同様に$\boldsymbol{x}_{k+1}^{+}$はクォータニオンの要件である大きさ1を満たしていない可能性があるので,正規化する.

ところで$\boldsymbol{P}^{+}$が対角行列だとすると
\begin{align}
\boldsymbol{P}_{k+1}^{-} &= \boldsymbol{F}_{k} \boldsymbol{P}_{k+1}^{+} \boldsymbol{F}_{k}^{T} +
  \boldsymbol{Q}_{k+1}\\
  &=
  {\tiny
    \begin{pmatrix}
      \Box \clubsuit \Box^{T} + \triangle \spadesuit \triangle^{T} + \boldsymbol{Q}_{k+1}^{upper} & \triangle \spadesuit\\
      \spadesuit \triangle^{T} & \spadesuit + \boldsymbol{Q}_{k+1}^{lower}
    \end{pmatrix}
  }\\
  &=
  \begin{pmatrix}
    \clubsuit^{'} & \triangle \spadesuit\\
    \spadesuit \triangle^{T} & \spadesuit + \boldsymbol{Q}_{k+1}^{lower}
  \end{pmatrix}\\
  \boldsymbol{P}_{k+1}^{+} &=
  {\tiny
    \begin{pmatrix}
      \clubsuit^{'} - \clubsuit^{'} \pounds^{T} \left(\pounds \clubsuit^{'} \pounds^{T} + \boldsymbol{R}_{k+1}\right)^{-1} \pounds \clubsuit^{'} & \triangle \spadesuit\\
      \spadesuit \triangle^{T} - \spadesuit \triangle^{T} \pounds^{T} \left(\pounds \clubsuit^{'} \pounds^{T} + \boldsymbol{R}_{k+1}\right)^{-1} \pounds \clubsuit^{'} & \spadesuit + \boldsymbol{Q}_{k+1}^{lower}
    \end{pmatrix}
  }
\end{align}
となるので,対角行列ではないっぽい.

ところで,$\boldsymbol{Q}$と$\boldsymbol{R}$の決め方としては,やはりセンサの信頼度を入れるのが正解っぽい.
$\boldsymbol{R}$はそのままセンサの分散を入れれば良いが,$\boldsymbol{Q}$が問題.
一説によると
\begin{align}
  \boldsymbol{V}_{k+1} &= \frac{\partial f}{\partial \boldsymbol{u}_{k+1}}\\
  \boldsymbol{Q}_{k+1} &=
  \boldsymbol{V}_{k+1}
  \begin{pmatrix}
    \sigma_{\omega_x}^2 & 0 & 0\\
    0 & \sigma_{\omega_y}^2 & 0\\
    0 & 0 & \sigma_{\omega_x}^2
  \end{pmatrix}
  \boldsymbol{V}_{k+1}^{T}\\
  \boldsymbol{R}_{k+1} &=
  \begin{pmatrix}
    \sigma_{acc_x}^2 & 0 & 0\\
    0 & \sigma_{acc_y}^2 & 0\\
    0 & 0 & \sigma_{acc_x}^2
  \end{pmatrix}
\end{align}
とするのが良い?
こうすると,$\boldsymbol{Q}$の右下は$\boldsymbol{O}$になる.

\begin{align}
  \boldsymbol{V}_{k+1} &= \frac{\partial f}{\partial \boldsymbol{u}_{k+1}}\\
  &=
  \begin{pmatrix}
    -\tmpbibun{1} & -\tmpbibun{2} & -\tmpbibun{3}\\
     \tmpbibun{0} & -\tmpbibun{3} &  \tmpbibun{2}\\
     \tmpbibun{3} &  \tmpbibun{0} & -\tmpbibun{1}\\
    -\tmpbibun{2} &  \tmpbibun{1} &  \tmpbibun{0}\\
    0 & 0 & 0\\
    0 & 0 & 0\\
    0 & 0 & 0
  \end{pmatrix}\\
  &=
  \begin{pmatrix}
    \boldsymbol{V}_{k+1}^{upper}\\
    \boldsymbol{O}
  \end{pmatrix}\\
  \boldsymbol{Q}_{k+1}
  &=
  \boldsymbol{V}_{k+1}
  \boldsymbol{\Sigma}_{\boldsymbol{\omega}}
  \boldsymbol{V}_{k+1}^{T}\\
  &=
  \begin{pmatrix}
    \boldsymbol{V}_{k+1}^{upper} \boldsymbol{\Sigma}_{\boldsymbol{\omega}} {\boldsymbol{V}_{k+1}^{upper}}^{T} & \boldsymbol{O}\\
    \boldsymbol{O} & \boldsymbol{O}
  \end{pmatrix}
\end{align}

\subsection{autopilotの実装}
\url{http://autopilot.sourceforge.net/kalman.html}がwebページでは紹介されているけど,実装はもっと別のことをしているっぽい.
ジャイロが2つしかついていないっぽいのと,$C$の計算が謎い.

\begin{align}
  \boldsymbol{A} =
  \begin{pmatrix}
    0 & -p & -q & 0 & \frac{q_1}{2} & \frac{q_2}{2}\\
    p & 0 & 0 & -q & -\frac{q_0}{2} & \frac{q_3}{2}\\
    q & 0 & 0 & p & -\frac{q_3}{2} & -\frac{q_0}{2}\\
    0 & q & -p & 0 & \frac{q_2}{2} & -\frac{q_1}{2}\\
    0 & 0 & 0 & 0 & 0 & 0\\
    0 & 0 & 0 & 0 & 0 & 0
  \end{pmatrix}\\
  \boldsymbol{Q} =
  \begin{pmatrix}
    0.01^2\\
    & 0.01^2\\
    && 0.01^2\\
    &&& 0.01^2\\
    &&&& 0.03\\
    &&&&& 0.03
  \end{pmatrix}\\
  \boldsymbol{P}^{-}_{k+1} = \boldsymbol{P}^{+}_{k} + \Delta t \left(\boldsymbol{Q} + \boldsymbol{A}_{k+1} \boldsymbol{P}^{+}_{k} + \boldsymbol{P}^{+}_{k} \boldsymbol{A}_{k+1}^{T} \right)\\
  \tilde{\boldsymbol{q}}_{k+1}^{-} = \tilde{\boldsymbol{q}}_{k}^{+} +
  \Delta t
  \begin{pmatrix}
    0 & -p & -q & 0\\
    p & 0 & 0 & -q\\
    q & 0 & 0 & p\\
    0 & q & -p & 0
  \end{pmatrix}
  \tilde{\boldsymbol{q}}_{k}^{+}
\end{align}

\subsection{Discrete-time extended Kalman filter}
\url{https://en.wikipedia.org/wiki/Extended_Kalman_filter#Discrete-time_extended_Kalman_filter} によると,わざわざ自分で離散系に直さなくても良いらしい.

\begin{align}
  \dot{\boldsymbol{x}}_{t}
  &=
  f(\boldsymbol{x}_{t}, \boldsymbol{u}_{t})\\
  &=
  \begin{pmatrix}
    \dot{\tilde{\boldsymbol{q}}}_{t}\\
    \dot{\boldsymbol{b}}_{\boldsymbol{\omega}_t}
  \end{pmatrix}\\
  &=
  \begin{pmatrix}
    \frac{1}{2}
    \boldsymbol{\Omega}(\boldsymbol{\omega}_t - \boldsymbol{b}_{\boldsymbol{\omega}_t})
    \tilde{\boldsymbol{q}}_{t}\\
    0\\
    0\\
    0
  \end{pmatrix}
\end{align}
より
\begin{align}
  \boldsymbol{F}_{t}
  &=
  \left. \frac{\partial f}{\partial \boldsymbol{x}}\right|_{\boldsymbol{x}_{t}, \boldsymbol{u}_{t}}\\
  &=
  \begin{pmatrix}
    \multicolumn{4}{c}{\multirow{4}{*}{$\frac{1}{2} \boldsymbol{\Omega}(\boldsymbol{\omega}_t - \boldsymbol{b}_{\boldsymbol{\omega}_t})$}} &  \frac{q_1}{2} &  \frac{q_2}{2} & \frac{q_3}{2}\\
    \multicolumn{4}{c}{} & -\frac{q_0}{2} &  \frac{q_3}{2} & -\frac{q_2}{2}\\
    \multicolumn{4}{c}{} & -\frac{q_3}{2} & -\frac{q_0}{2} &  \frac{q_1}{2}\\
    \multicolumn{4}{c}{} &  \frac{q_2}{2} & -\frac{q_1}{2} &  \frac{q_0}{2}\\
    \multicolumn{7}{c}{\multirow{3}{*}{$\boldsymbol{O}$}}\\
    \multicolumn{7}{c}{}\\
    \multicolumn{7}{c}{}
  \end{pmatrix}
\end{align}
となって
\begin{equation}
  \dot{\boldsymbol{P}}_{t} = \boldsymbol{F}_{t} \boldsymbol{P}_{t} + \boldsymbol{P}_{t} \boldsymbol{F}_{t}^{T} + \boldsymbol{Q}_{t}
\end{equation}

これは結局普通のEKFとほぼ同じなんだけど,
\begin{itemize}
  \item $(\Delta t)^2$の項が消えている
  \item $Q$が微分の式にかける
\end{itemize}
なので,こちらの方が効率良さそう.
