\section{オイラー角表現のKalmanFilter} \label{sec:rpy_kalman_filter}
ここでいうオイラー角はZXYオイラー角で,ロール・ピッチ・ヨーのこと.

IMUが観測する角速度とRPYの関係式は
\begin{equation}
  \begin{pmatrix}
    \dot{\alpha}\\
    \dot{\beta}\\
    \dot{\gamma}
  \end{pmatrix} =
  \begin{pmatrix}
    1 & \frac{\sin \alpha \sin \beta}{\cos \beta} & \frac{\cos \alpha \sin \beta}{\cos \beta} \\
    0 & \cos \alpha & - \sin \alpha \\
    0 & \frac{\sin \alpha}{\cos \beta} & \frac{\cos \alpha}{\cos \beta}
  \end{pmatrix}
  \begin{pmatrix}
    \omega_x\\
    \omega_y\\
    \omega_z
  \end{pmatrix}\label{eq:rpy-gyro}
\end{equation}
で,自身が加速していない場合の加速度と姿勢の関係式は
\begin{align}
  \begin{pmatrix}
    acc_x\\
    acc_y\\
    acc_z
  \end{pmatrix} &= {}^{W}_{A}\boldsymbol{R}^{T} \boldsymbol{g}^{W}\\
  &= \begin{pmatrix}
    -\sin \beta\\
    \sin \alpha \cos \beta\\
    \cos \alpha \cos \beta
  \end{pmatrix}
  g
\end{align}

\autoref{eq:rpy-gyro}を$\alpha = 0$,$\beta = 0$で近似すれば線形+互いに素にできるし,
近似しなければEKFでできる.
EKFの場合は$\beta = \frac{\pi}{2}$で\autoref{eq:rpy-gyro}が0除算してしまって破綻する.

\subsection{線形カルマンフィルタ}
\subsubsection{事前準備}
関係式(ゼロ点付近で近似)
\begin{equation}
  \begin{pmatrix}
    \dot{\alpha}\\
    \dot{\beta}\\
    \dot{\gamma}
  \end{pmatrix} =
  \begin{pmatrix}
    1 & 0 & 0 \\
    0 & 1 & 0 \\
    0 & 0 & 1
  \end{pmatrix}
  \begin{pmatrix}
    \omega_x\\
    \omega_y\\
    \omega_z
  \end{pmatrix}
\end{equation}
内部状態(オイラー角+その微分(角速度)のバイアス)
\begin{equation}
  \boldsymbol{x} =
  \begin{pmatrix}
    \alpha\\
    \beta\\
    \gamma\\
    b_{w_x}\\
    b_{w_y}\\
    b_{w_z}
  \end{pmatrix}
\end{equation}
予測(互いに独立しているので,decouple可能)
{\tiny
  \begin{equation}
    \boldsymbol{x}_{k+1}^{-} =
    \begin{pmatrix}
      1 & 0 & 0 & -\Delta t & 0 & 0\\
      0 & 1 & 0 & 0 & -\Delta t & 0\\
      0 & 0 & 1 & 0 & 0 & -\Delta t\\
      0 & 0 & 0 & 1 & 0 & 0\\
      0 & 0 & 0 & 0 & 1 & 0\\
      0 & 0 & 0 & 0 & 0 & 1
    \end{pmatrix}
    \boldsymbol{x}_{k}^{+} +
    \begin{pmatrix}
      \Delta t & 0 & 0\\
      0 & \Delta t & 0\\
      0 & 0 & \Delta t\\
      0 & 0 & 0\\
      0 & 0 & 0\\
      0 & 0 & 0
    \end{pmatrix}
    \begin{pmatrix}
      \dot{\alpha}\\
      \dot{\beta}\\
      \dot{\gamma}
    \end{pmatrix}
  \end{equation}
}
観測(線形にするために観測した加速度を元に求めた姿勢を観測したことにする)
\begin{align}
  \alpha &= \arctan \frac{\sin \alpha}{\cos \alpha} = \arctan \frac{acc_y}{acc_z}\\
  \beta &= \arctan \frac{\sin \beta}{\cos \beta} = \arctan \frac{-acc_x}{\sqrt{acc_y^2 + acc^2_z}}
\end{align}

\subsubsection{中身}
\begin{align}
  \begin{pmatrix}
    \alpha_{k+1}^{-}\\
    b_{{\omega_x}_{k+1}}^{-}
  \end{pmatrix} =
  \begin{pmatrix}
    1 & -\Delta t\\
    0 & 1
  \end{pmatrix}
  \begin{pmatrix}
    \alpha_{k}^{+}\\
    b_{{\omega_x}_{k}}^{+}
  \end{pmatrix} +
  \begin{pmatrix}
    \Delta t\\
    0
  \end{pmatrix}
  \dot{\alpha}_{k+1}
\end{align}
\begin{align}
  \boldsymbol{P}_{k+1}^{-}
  &=
  \begin{pmatrix}
    1 & -\Delta t\\
    0 & 1
  \end{pmatrix}
  \boldsymbol{P}_{k+1}^{+}
  \begin{pmatrix}
    1 & -\Delta t\\
    0 & 1
  \end{pmatrix}^{T} +
  \boldsymbol{Q}_{k+1}\\
  &=
  \begin{pmatrix}
    1 & -\Delta t\\
    0 & 1
  \end{pmatrix}
  \begin{pmatrix}
    \clubsuit & \heartsuit\\
    \diamondsuit & \spadesuit
  \end{pmatrix}
  \begin{pmatrix}
    1 & -\Delta t\\
    0 & 1
  \end{pmatrix}^{T} +
  \boldsymbol{Q}_{k+1}\\
  &=
  \begin{pmatrix}
    \clubsuit - \Delta t \diamondsuit & \heartsuit - \Delta t \spadesuit\\
    \diamondsuit & \spadesuit
  \end{pmatrix}
  \begin{pmatrix}
    1 & 0\\
    -\Delta t & 1
  \end{pmatrix} +
  \boldsymbol{Q}_{k+1}\\
  &=
  \begin{pmatrix}
    \clubsuit - (\diamondsuit + \heartsuit) \Delta t + \spadesuit \Delta^2 t & \heartsuit - \spadesuit \Delta t\\
    \diamondsuit - \spadesuit \Delta t & \spadesuit
  \end{pmatrix} +
  \boldsymbol{Q}_{k+1}\\
  &=
  {\footnotesize
    \begin{pmatrix}
      \clubsuit - (\diamondsuit + \heartsuit) \Delta t + \spadesuit \Delta^2 t + Q^{upper}_{k+1} & \heartsuit - \spadesuit \Delta t\\
      \diamondsuit - \spadesuit \Delta t & \spadesuit + Q^{lower}_{k+1}
    \end{pmatrix}
  }
\end{align}
加速度から求めた姿勢を観測したことになっているので,
\begin{equation}
  h(\begin{pmatrix}
    \alpha_{k+1}^{-}\\
    b_{{\omega_x}_{k+1}}^{-}
  \end{pmatrix}) =
  \begin{pmatrix}
    1 & 0
  \end{pmatrix}
  \begin{pmatrix}
    \alpha_{k+1}^{-}\\
    b_{{\omega_x}_{k+1}}^{-}
  \end{pmatrix} =
  \alpha_{k+1}^{-}
\end{equation}

よって,
\begin{align}
  y_{k+1} &= z_{k+1} - \alpha_{k+1}^{-}\\
  S_{k+1} &=
  \begin{pmatrix}
    1 & 0
  \end{pmatrix}
  \boldsymbol{P}_{k+1}^{-}
  \begin{pmatrix}
    1 & 0
  \end{pmatrix}^{T} + R_{k+1}\\
  &=
  \clubsuit - (\diamondsuit + \heartsuit) \Delta t + \spadesuit \Delta^2 t + Q^{upper}_{k+1} + R_{k+1}\\
  \boldsymbol{K}_{k+1} &=
  \boldsymbol{P}_{k+1}^{-}
  \begin{pmatrix}
    1 & 0
  \end{pmatrix}^{T}
  S_{k+1}^{-1}\\
  &=
  \begin{pmatrix}
    \clubsuit - (\diamondsuit + \heartsuit) \Delta t + \spadesuit \Delta^2 t + Q^{upper}_{k+1}\\
    \diamondsuit - \spadesuit \Delta t
  \end{pmatrix}
  S_{k+1}^{-1}\\
  &=
  \begin{pmatrix}
    \frac{1}{1 + \frac{R_{k+1}}{\clubsuit - (\diamondsuit + \heartsuit) \Delta t + \spadesuit \Delta^2 t + Q^{upper}_{k+1}}}\\
    \frac{\diamondsuit - \spadesuit \Delta t}{\clubsuit - (\diamondsuit + \heartsuit) \Delta t + \spadesuit \Delta^2 t + Q^{upper}_{k+1} + R_{k+1}}
  \end{pmatrix}
\end{align}
となるので
\begin{align}
  \begin{pmatrix}
    \alpha_{k+1}^{+}\\
    b_{{\omega_x}_{k+1}}^{+}
  \end{pmatrix} &=
  \begin{pmatrix}
    \alpha_{k+1}^{-}\\
    b_{{\omega_x}_{k+1}}^{-}
  \end{pmatrix} +
  \boldsymbol{K}_{k+1} y_{k+1}\\
  &=
  \begin{pmatrix}
    \frac{{\color{red} \clubsuit - (\diamondsuit + \heartsuit) \Delta t + \spadesuit \Delta^2 t + Q^{upper}_{k+1}} z_{k+1} + {\color{blue} R_{k+1}} \alpha_{k+1}^{-}}{{\color{red} \clubsuit - (\diamondsuit + \heartsuit) \Delta t + \spadesuit \Delta^2 t + Q^{upper}_{k+1}} + {\color{blue} R_{k+1}}}\\
    b_{{\omega_x}_{k+1}}^{+}
  \end{pmatrix}\\
  \boldsymbol{P}_{k+1}^{+} &= \left(\boldsymbol{E} - \boldsymbol{K}_{k+1}
  \begin{pmatrix}
    1 & 0
  \end{pmatrix}
  \right) \boldsymbol{P}_{k+1}^{-}\\
  &=
  \left(\boldsymbol{E} -
  {\tiny
    \begin{pmatrix}
      \frac{1}{1 + \frac{R_{k+1}}{\clubsuit - (\diamondsuit + \heartsuit) \Delta t + \spadesuit \Delta^2 t + Q^{upper}_{k+1}}} & 0\\
      \frac{\diamondsuit - \spadesuit \Delta t}{\clubsuit - (\diamondsuit + \heartsuit) \Delta t + \spadesuit \Delta^2 t + Q^{upper}_{k+1} + R_{k+1}} & 0
    \end{pmatrix}
  }
  \right) \boldsymbol{P}_{k+1}^{-}\\
  &=
  \begin{pmatrix}
    \frac{1}{1 + \frac{\clubsuit - (\diamondsuit + \heartsuit) \Delta t + \spadesuit \Delta^2 t + Q^{upper}_{k+1}}{R_{k+1}}} & 0\\
    -\frac{\diamondsuit - \spadesuit \Delta t}{\clubsuit - (\diamondsuit + \heartsuit) \Delta t + \spadesuit \Delta^2 t + Q^{upper}_{k+1} + R_{k+1}} & 1
  \end{pmatrix}
  \boldsymbol{P}_{k+1}^{-}
\end{align}
となる.

ここで,2016/02/22のhrpsysの実装では
\begin{align}
  \begin{pmatrix}
    \alpha_{0}^{+}\\
    b_{{\omega_x}_{0}}^{+}
  \end{pmatrix} &=
  \begin{pmatrix}
    0\\
    0
  \end{pmatrix}\\
  \boldsymbol{P}_{0}^{+} &= \boldsymbol{O}\\
  \boldsymbol{Q} &=
  \begin{pmatrix}
    0.001 * \Delta t & 0\\
    0 & 0.003 * \Delta t
  \end{pmatrix}\\
  R &= 1000
\end{align}
なので,1ステップ進めると
\begin{align}
  \begin{pmatrix}
    \alpha_{1}^{-}\\
    b_{{\omega_x}_{1}}^{-}
  \end{pmatrix} &=
  \begin{pmatrix}
    {\omega_x}_{1} \Delta t\\
    0
  \end{pmatrix}\\
  \boldsymbol{P}_{1}^{-} &=
  \begin{pmatrix}
    0.001 * \Delta t & 0\\
    0 & 0.003 * \Delta t
  \end{pmatrix}\\
  S_{1} &= 0.001 * \Delta t + 1000\\
  \boldsymbol{K}_{1} &=
  \begin{pmatrix}
    \frac{0.001 * \Delta t}{0.001 * \Delta t + 1000}\\
    0
  \end{pmatrix}\\
  \begin{pmatrix}
    \alpha_{1}^{+}\\
    b_{{\omega_x}_{1}}^{+}
  \end{pmatrix} &=
  \begin{pmatrix}
    {\omega_x}_{1} \Delta t\\
    0
  \end{pmatrix} +
  \begin{pmatrix}
    \frac{0.001 * \Delta t}{0.001 * \Delta t + 1000}\\
    0
  \end{pmatrix}
  \left(z_{1} - {\omega_x}_{1} \Delta t\right)\\
  &=
  \begin{pmatrix}
    \frac{0.001 * \Delta t * z_{1} + 1000 * {\omega_x}_{1} \Delta t}{0.001 * \Delta t + 1000}\\
    0
  \end{pmatrix}\\
  \boldsymbol{P}_{1}^{+} &=
  \begin{pmatrix}
    \frac{1000}{0.001 * \Delta t + 1000} & 0\\
    0 & 1
  \end{pmatrix}
  \begin{pmatrix}
    0.001 * \Delta t & 0\\
    0 & 0.003 * \Delta t
  \end{pmatrix}\\
  &=
  \begin{pmatrix}
    \frac{\Delta t}{0.001 * \Delta t + 1000} & 0\\
    0 & 1
  \end{pmatrix}\\
  &\simeq
  \begin{pmatrix}
    0.001 * \Delta t & 0\\
    0 & 1
  \end{pmatrix}
\end{align}
で,もう1ステップ進めると
\begin{align}
  \begin{pmatrix}
    \alpha_{2}^{-}\\
    b_{{\omega_x}_{2}}^{-}
  \end{pmatrix} &=
  \begin{pmatrix}
    \frac{0.001 * \Delta t * z_{1} + 1000 * {\omega_x}_{1} \Delta t}{0.001 * \Delta t + 1000} + {\omega_x}_2 \Delta t\\
    0
  \end{pmatrix}\\
  \boldsymbol{P}_{2}^{-} &=
  {\footnotesize
    \begin{pmatrix}
      \frac{\Delta t}{0.001 * \Delta t + 1000} + \Delta^2 t + 0.001 * \Delta t & - \Delta t\\
      - \Delta t & 1 + 0.003 * \Delta t
    \end{pmatrix}}\\
  &\simeq
  \begin{pmatrix}
    \frac{\left(2 + 1000 \Delta t\right) e^{3} \Delta t}{\Delta t + 1e^{6}} & - \Delta t\\
    - \Delta t & 1 + 0.003 * \Delta t
  \end{pmatrix}\\
  S_{2} &\simeq \frac{\left(2 + 1000 \Delta t\right)\Delta t}{0.001 * \Delta t + 1000} + 1000\\
  &= \frac{\left(3 + 1000 \Delta t\right)\Delta t + 1e^{6}}{0.001 * \Delta t + 1000}\\
  \boldsymbol{K}_{2}
  &\simeq
  \begin{pmatrix}
    \frac{\left(2 + 1000 \Delta t\right)\Delta t}{\left(3 + 1000 \Delta t\right)\Delta t + 1e^{6}}\\
    \frac{-0.001 * \Delta^2 t - 1000 \Delta t}{\left(3 + 1000 \Delta t\right)\Delta t + 1e^{6}}
  \end{pmatrix}\\
  &\simeq
  \begin{pmatrix}
    \frac{\left(2 + 1000 \Delta t\right)\Delta t}{\left(3 + 1000 \Delta t\right)\Delta t + 1e^{6}}\\
    \frac{- 1000 \Delta t}{\left(3 + 1000 \Delta t\right)\Delta t + 1e^{6}}
  \end{pmatrix}\\
  \begin{pmatrix}
    \alpha_{2}^{+}\\
    b_{{\omega_x}_{2}}^{+}
  \end{pmatrix} &=
  \begin{pmatrix}
    \alpha_{2}^{-}\\
    b_{{\omega_x}_{2}}^{-}
  \end{pmatrix} + \boldsymbol{K}_{2} \left(z_2 - \alpha_{2}^{-}\right)\\
  &=
  \begin{pmatrix}
    \frac{\left(2 + 1000 \Delta t\right)\Delta t z_2 + \left(\Delta t + 1e^6\right) \alpha_2}{\left(3 + 1000 \Delta t\right)\Delta t + 1e^{6}}\\
    0
  \end{pmatrix}\\
  \boldsymbol{P}_{2}^{+} &=
  {\tiny
    \setlength\arraycolsep{3pt}
    \begin{pmatrix}
      \frac{\Delta t + 1e^6}{\left(3 + 1000 \Delta t\right)\Delta t + 1e^{6}} & 0\\
      \frac{1000 \Delta t}{\left(3 + 1000 \Delta t\right)\Delta t + 1e^{6}} & 1
    \end{pmatrix}
    \begin{pmatrix}
      \frac{\left(2 + 1000 \Delta t\right) e^{3} \Delta t}{\Delta t + 1e^{6}} & - \Delta t\\
      - \Delta t & 1 + 0.003 * \Delta t
    \end{pmatrix}}\\
  &=
  \begin{pmatrix}
    \frac{\left(2 + 1000 \Delta t\right) e^{3} \Delta t}{\left(3 + 1000 \Delta t\right)\Delta t + 1e^{6}} & \frac{-\left(\Delta t + 1e^6\right) \Delta t}{\left(3 + 1000 \Delta t\right)\Delta t + 1e^{6}}\\
    ? & ?
  \end{pmatrix}\\
  &\simeq
  \begin{pmatrix}
    0.002 * \Delta t & ?\\
    ? & ?
  \end{pmatrix}
\end{align}
となって,最終的には
\begin{align}
  \boldsymbol{P}^{+} &=
  \begin{pmatrix}
    0.558269 & -0.077438\\
    -0.077438 & 0.0216277
  \end{pmatrix}\\
  \boldsymbol{K} &=
  \begin{pmatrix}
    0.000558269\\
    -7.7438e-05
  \end{pmatrix}\\
  S &= 1000.56
\end{align}
に収束していた.なんでだろう.

ただ,とりあえずジャイロの積算のみに頼っていることが分かったので,
ゼロ点がずれていると一生ずれるのと早く動くとずれるということがわかった.

\autoref{eq:rpy-gyro}を使って入力しているジャイロを変換することで,
ゼロ点周りでの近似誤差を減らすことが可能.

\subsection{非線形カルマンフィルタ}
